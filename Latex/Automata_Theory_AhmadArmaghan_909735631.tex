\documentclass[10.5pt]{article}
\title{Introduction to Automata Theory,\\Formal Languages and\\Computation}
\author{Shyamalendu Kandar\\\\Latex by Ahmad Armaghan\\StudentId : 909735631\\Professor: Mr. Ali Razavi Ebrahimi\\Course: Languages Theory 1115157-01}
\date{\today}
\usepackage{tabto}
\usepackage{graphicx}
\usepackage{tabularx}
\usepackage{fancyhdr}
\usepackage{wrapfig}
\usepackage{circuitikz}
\usepackage{blindtext}
\usepackage{geometry}
 \geometry{
 a4paper,
 total={170mm,257mm},
 left=20mm,
 top=20mm,
 }
\usepackage{multicol}
\setlength{\columnsep}{1cm}
\setcounter{page}{8}
\pagestyle{fancy}
\fancyhf{}
\fancyhead[LE,RO]{}
\fancyhead[LO,RE]{}
\fancyfoot[CE,CO]{}
\fancyfoot[LE,RO]{}
\rhead{\color{gray} Basic Terminology \textbar{} \thepage}
\lfoot{\tiny \color{gray} By Ahmad Armaghan ,Student ID : 909735631 , Course : Language theory, Professor : Mr. Ali Razavi Ebrahimi}
\renewcommand{\headrulewidth}{0pt}
\renewcommand{\footrulewidth}{1pt}
\renewcommand{\footrule}{\hbox to\headwidth{\color{gray}\leaders\hrule height \footrulewidth\hfill}}
\setlength{\headheight}{15pt}

\begin{document}
\maketitle
\clearpage
\begin{multicols}{2}
[
\noindent device. As gates are digital components, all these voltage ranges are restricted to two values ‘high’ and
‘low’. In digital electronics, ‘high’ is represented as ‘1’ and low is represented as ‘0’. The electronic gate
acts as a switching device which either permits fl ow of current or blocks it. The basic gates are AND,
OR, NOT, and XOR.
]
\paragraph{}
The AND gate has two or more inputs and a single
output (Fig. 1.9). If all of the inputs are ‘1’ the
output is ‘1’; otherwise, the output is ‘0’. For a twoinput
AND gate, with input labels ‘A’ and ‘B’, the
output function is written as T = AB. A two-input
AND gate is represented by the following symbol
and the truth table is as follows.
\paragraph{}
The OR gate also has two or more inputs and a
single output (Fig. 1.10). If any of the inputs is ‘1’,
the output is ‘1’; otherwise, it is ‘0’. For a two-input
OR gate, with input labels ‘A’ and ‘B’, the output
function is written as T = A + B. A two-input OR
gate is represented by the following symbol and the
truth table is as follows.
\paragraph{}
The NOT gate has a single input and a single
output (Fig. 1.11). The function of the NOT gate is
to reverse the input. The output function of the NOT
gate is represented as T = A, where ‘A’ is the input.
The symbol and truth table of the NOT gate is given
as follows.
\paragraph{}
The exclusive-OR, in short XOR, gate is a complex
gate (Fig. 1.12). A two-input XOR gate gives
the output ‘1’ when its two inputs are different and
the output ‘0’ when the inputs are the same. For two
inputs ‘A’ and ‘B’, the output functions are represented
by $T = A \oplus B = AB + AB$
\columnbreak
\begin{wrapfigure}{!h}{0.4\textwidth} 
	\begin{center}
		\begin{circuitikz} \draw
			(0,2) node[and port] (a) {};
		\end{circuitikz} 
		\begin{tabular}{ |c|c|c| } 
			\hline
			A & B & O/P \\
			\hline
			0 & 0 & 0 \\
			\hline
			0 & 1 & 0 \\
			\hline
			1 & 1 & 1 \\
			\hline
			1 & 0 & 0 \\
			\hline
		\end{tabular}
		\caption{\textit{The AND Gate with Truth Table}}
		\end{center}
\end{wrapfigure} 
\begin{wrapfigure}{!h}{0.4\textwidth}
	\begin{center}		
		\begin{circuitikz} \draw
			(0,2) node[or port] (a) {};
		\end{circuitikz} 
		\begin{tabular}{ |c|c|c| } 
			\hline
			A & B & O/P \\
			\hline
			0 & 0 & 0 \\
			\hline
			0 & 1 & 1 \\
			\hline
			1 & 1 & 1 \\
			\hline
			1 & 0 & 1 \\
			\hline
		\end{tabular}
		\caption{\textit{The OR Gate with Truth Table}}
	\end{center}
\end{wrapfigure} 
\begin{wrapfigure}{!h}{0.4\textwidth}
	\begin{center}		
		\begin{circuitikz} \draw
			(0,2) node[not port] (a) {};
		\end{circuitikz} 
		\begin{tabular}{ |c|c| } 
			\hline
			A & O/P \\
			\hline
			0 & 1 \\
			\hline
			1 & 0 \\
			\hline
		\end{tabular}
	\caption{\textit{The NOT Gate with Truth Table}}
	\end{center}
\end{wrapfigure}
\begin{wrapfigure}{!h!}{0.4\textwidth}
	\begin{center}		
		\begin{circuitikz} \draw
			(0,2) node[xor port] (a) {};
		\end{circuitikz} 
		\begin{tabular}{ |c|c|c| } 
			\hline
			A & B & O/P \\
			\hline
			0 & 0 & 0 \\
			\hline
			0 & 1 & 1 \\
			\hline
			1 & 1 & 0 \\
			\hline
			1 & 0 & 1 \\
			\hline
		\end{tabular}
	\caption{\textit{The XOR Gate with Truth Table}}
	\end{center}
\end{wrapfigure}
\begin{wrapfigure}{!h!}{0.0\textwidth}
		 
\end{wrapfigure}
\end{multicols}
\begin{center}
\begin{enumerate}
		\item[1.6] \textbf{Digital Circuit}
		\paragraph{}
A digital circuit is a circuit using logic gates where the signal must be one of two discrete levels. Each
level is interpreted as one of two different states depending on the voltage level (on/off, 0/1 or true/
false). The digital circuit is operated by the logic of the Boolean algebra. This logic is the foundation of
digital electronics and computer processing.
\paragraph{}
Depending on the output function, digital circuits are divided into two  groups:
		\item[1] \textbf{Combinational circuits:}
 The circuits where the output depends only on the present input, i.e.,
output is the function of only the present input, are called combinational circuits.
   \item[2]
   \textbf{Sequential circuit:} The circuits where the output depends on the external input and the stored
information at that time, i.e., output is the function of external input and the present stored information,
are called sequential circuits.
O/P = Func.(External I/P and Present stored information)
\end{enumerate}
\end{center}

\clearpage
\paragraph{}
The difference between the sequential and combinational circuit is that sequential circuit has memory
in the form of fl ip fl op, whereas combinational circuit does not have the memory element. A general
block diagram of sequential circuit is shown in Fig. 1.13.
\begin{figure}[h!]
	\begin{center}	
		\includegraphics[width=0.8\linewidth]{./Images/fig-1.13.PNG}
			\caption{\textit{Block Diagram of Sequential Circuit}}
  	\end{center}
\end{figure}

Sequential circuits fall into two classes: synchronous and asynchronous. Synchronization is usually
achieved by some timing device, such as clock. A clock produces equally spaced pulses. These pulses are fed into a circuit in such a way that various operations
of the circuit take place with the arrival of appropriate
clock pulses. Generally, the circuits, whose operations are
controlled by clock pulses, are called synchronous circuit.
\begin{wrapfigure}{!h!}{0.5\textwidth}
	\begin{center}	
		\includegraphics[width=0.9\linewidth]{./Images/fig-1.14.PNG}
			\caption{\textit{Synchronous Sequential Circuit}}
  	\end{center}
\end{wrapfigure}

\paragraph{}
The operation of an asynchronous circuit does not
depend on clock pulses. The operations in an asynchronous
circuit are controlled by a number of completion
and initialization signals. Here, the completion of one
operation is the initialization of the execution of the next
consecutive operation. The following (Fig. 1.14) block
diagram is that of a synchronous sequential circuit.
\paragraph{}
A synchronous sequential machine has fi nite number
of inputs. If a machine has n number of input variables,
the input set consists of 2n distinct inputs called input
alphabet I.
In the fi gure, the input alphabet is I = $\{I1, I2, ……, Ip\}$.
The number of outputs of a synchronous sequential
machine is also fi nite.

\paragraph{}
A synchronous sequential circuit can be designed by the following process:\\
\indent\textbullet{}{}From the problem description, design a state table and a state diagram (whichever is fi rst applicable).\\
\indent\textbullet{}{}Make the state table redundant by machine minimization. This removes some states and makes the
table redundant.\\
\indent\textbullet{}{}Perform state assignment by assigning the states to binary numbers. n binary numbers can assign
$2^{n}$ states.\\
\indent\textbullet{}{}After performing the state assignment, derive a transition table and an output table.\\
\indent\textbullet{}{}Derive the transitional function and the output function from the transitional table and the output
table.\\
\indent\textbullet{}Draw the circuit diagram.\\

\clearpage
\noindent The following examples design some synchronous sequential circuit.
\begin{enumerate}
		\item[Example 1.6] \textbf{Design a sequential circuit which performs the following:}
\end{enumerate}

\begin{center}
	\begin{tabular}{ c c c c } 		
		\hline
		A & B & O/P & Carry \\
		\hline
		0 & 0 & 0 & 0\\
		\hline
		0 & 1 & 1 & 0\\
		\hline
		1 & 0 & 1 & 0\\
		\hline
		1 & 1 & 0 & 1\\
		\hline
	\end{tabular}
\end{center}
\begin{wrapfigure}{!h!}{0.5\textwidth}
	\begin{center}	
		\includegraphics[width=0.9\linewidth]{./Images/fig-1.15.PNG}
			\caption{}
  	\end{center}
\end{wrapfigure}

\noindent The carry is added with the I/P’s in the next clock pulse.\\
\noindent \textbf{Solution:} Let us take two input strings $X_{1}=0111$ and $X_{2}=0101$.\\
\indent Here, the output at time $t_{i}$ is a function of the inputs $X_{1}$ and $X_{2}$ at the time $t_{i}$ and of the carry generated for the input at $t_{i-1}$.\\
O/P = func.(I/P at $t_{i}$ and the carry generated for
the input at $t_{i-1}$)\\
Therefore, this is a sequential circuit (Fig. 1.15).
If we look into the previous table, we will see
two types of cases arisen there. These are
\begin{enumerate}
		\item[1] The producing carry ‘0’
		\item[2] The producing carry ‘1’.
\end{enumerate}
\indent\indent We have to consider this as the O/P depends of the carry also.\\
\indent Let us take the cases as states. So, we can consider two states, A for (1) and B for (2). If we construct
a table for the inputs $X_{1}$ and $X_{2}$ by considering the states, it will become
\begin{center}
\noindent\rule{7.8cm}{1.2pt}\\
Next State, O/P(Z)
\end{center}
\begin{center}
	\begin{tabular}{ c c c c c} 		
		\hline
		Present State & $X_{1}X_{2} = 00$ & =01 & =11 & =10 \\
		\hline
		A & A,0 & A,1 & B,0 & A,1 \\
		\hline
		B & A,1 & B,0 & B,1 & B,0 \\
	\end{tabular}
\end{center}
\begin{center}
\noindent\rule{7.8cm}{1.2pt}\\
\end{center}
\noindent This type of table is called the state table.
\begin{wrapfigure}{!h!}{1\textwidth}
	\begin{center}	
		\includegraphics[width=0.8\linewidth]{./Images/fig-1.16.PNG}
			\caption{\textit{State Diagram for the Sequential Circuit}}
  	\end{center}
\end{wrapfigure}
This type of graph is called state graph or state diagram.
\clearpage
\paragraph{}
For designing a circuit, we need only ‘0’ and ‘1’, i.e., Boolean values. So, the states A and B must be
assigned to some Boolean number. As there are only two states A and B, only one-digit Boolean value
is suffi cient. Let us represent A as ‘0’ and B as ‘1’.\\
\indent By assigning these Boolean values to A and B, the modifi ed table becomes
\begin{center}
\noindent\rule{15.2 cm}{1.2pt}\\
\end{center}
\begin{center}
\begin{tabular}{ c c c}
\begin{tabular}{ c } {}\\ Present State(y) \\ \hline 0 \\ \hline 1\\ \end{tabular} & \begin{tabular}{ c c c c} Next State,(Y) \\ \hline $X_{1}X_{2}=00$ & =01 & =11 & =10  \\ \hline 0 & 0 & 1 & 0 \\ \hline 1 & 1 & 1 & 1\\  \end{tabular} & \begin{tabular}{ c c c c} O/P(Z) \\ \hline =00 & =01 & =11 & =10  \\ \hline 0 & 1 & 0 & 1\\ \hline 1 & 0 & 1 & 0\\  \end{tabular} \\
\end{tabular}
\end{center}
\begin{center}
\noindent\rule{15.2 cm}{2pt}\\
\end{center}
\noindent The function for next state \tab $Y=X_{1}X_{2} + X_{1y} + X_{2y}$\\
\noindent The function for output \tab $Z^\prime=X^\prime_{1}X^\prime_{2y} + X^\prime_{1}X_{2y^\prime} + X_{1}X^\prime_{2y} + X_{1}X_{2y} = X_{1} \oplus X_{2} \oplus y$.
\paragraph{}
From this function, the digital circuit is designed as denoted in Fig.1.17.
\begin{wrapfigure}{!h!}{1\textwidth}
	\begin{center}	
		\includegraphics[width=0.8\linewidth]{./Images/fig-1.17.PNG}
			\caption{\textit{Circuit Diagram for the Sequential Circuit}}			
  	\end{center}
  	\noindent This is the circuit for full binary adder.
\end{wrapfigure}
\clearpage
\end{document}
